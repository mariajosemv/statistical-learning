\begin{frame}{When to use Linear vs quadratic discriminant analysis?}



\begin{columns}[T]
\column{0.5\linewidth}
  \textbf{With linear discriminant analysis: }
    \begin{itemize}
        \item<1-> Assumes common covariance matrix and becomes linear in $x$, for a total of \textcolor{blue}{$Kp$} linear coefficients to estimate. 

        \item<3-> LDA is a much less flexible classifier. 

        \item<5-> LDA can suffer from high bias.

    \end{itemize}


    \column{0.5\linewidth}
   \textbf{With quadratic discriminant analysis:}
    \begin{itemize}
        \item<2->Estimates a separate covariance matrix for each class, for a total of \textcolor{blue}{$Kp(p+1)/2$} parameters. 

        \item<4-> Recommended if the training set is very large, so that the variance is not a major concern. 
        
        \item<6-> Use it when the assumption of a common covariance matrix is clearly untenable.
    \end{itemize}
\end{columns}

\end{frame}