\begin{frame}{Other considerations in the Regression model}{Removing the Additive Assumption}

Consider the standard linear regression model with two variables, 
        $$Y = \beta_0 + \beta_1 X_1 + \beta_2 + \epsilon$$
According to this, \pause
\begin{itemize}
    \item A one-unit increase in $X_1$ is associated with an average increase in $Y$ of $\beta_1$ units. \pause
    \item Notice that the presence of $X_2$ does not alter this statement. \pause
    \item We can extend this model to include an \textbf{interaction term} with $X_1$ and $X_2$. \pause 
    \begin{equation*}
        Y = \beta_0 + \beta_1 X_1 + \beta_2 X_2 + \beta_3 X_1 X_2 + \epsilon.
    \end{equation*}
    \item \pause Now $Y$ can be rewritten as, 
    \begin{align*}
        Y &=   \beta_0 + (\beta_1 X_1 + \beta_3 X_2)X_1 + \beta_2 X_2 + \epsilon \\
        Y &=   \beta_0 + \Tilde{\beta_1}X_1 + \beta_2 X_2 + \epsilon \\
        & \text{with } \Tilde{\beta_1} = \beta_1 + \beta_3 X_2. 
    \end{align*} 
\end{itemize}
    
\end{frame}

\begin{frame}{Other considerations in the Regression model}{Removing the Additive Assumption}

    \begin{align*}
        Y &=   \beta_0 + (\beta_1 X_1 + \beta_3 X_2)X_1 + \beta_2 X_2 + \epsilon \\
        Y &=   \beta_0 + \Tilde{\beta_1}X_1 + \beta_2 X_2 + \epsilon \\
         & \text{with } \Tilde{\beta_1} = \beta_1 + \beta_3 X_2.
    \end{align*}

\begin{itemize}
    \item Since $\Tilde{\beta_1}$ is now a function of $X_1$, the association between $X_1$ and $Y$ is no longer constant. \pause
    \item A change in the value of $X_2$ will change the association between $X_1$ and $Y$. \pause
    \item Similarly, a change in the value of $X_1$ changes the association between $X_2$ and $Y$.\pause
\end{itemize}
\end{frame}

\begin{frame}{Other considerations in the Regression model}{Removing the Additive Assumption}

    \begin{align*}
        Y &=   \beta_0 + (\beta_1 X_1 + \beta_3 X_2)X_1 + \beta_2 X_2 + \epsilon \\
        Y &=   \beta_0 + \Tilde{\beta_1}X_1 + \beta_2 X_2 + \epsilon \\
        & \text{with } \Tilde{\beta_1} = \beta_1 + \beta_3 X_2.
    \end{align*}
    

\begin{block}{The hierarchical principle}
        If we include an \textbf{interaction} in a model, we should also include the \textbf{main effects}, even if the p-values associated with principle their coefficients are not significant.
\end{block} \pause
Why? \pause
\begin{itemize}
    \item If $X_1 \times X_2$ is related to the response, then whether or not the coefficients of $X_1$ or $X_2$ are exactly zero is of little interest. \pause
    \item $X_1 \times X_2$ is typically correlated with $X_1$ and $X_2$, and so leaving them out tends to alter the meaning of the interaction. \pause
\end{itemize}
    
\end{frame}


    

