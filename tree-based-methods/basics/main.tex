\subsection{Prediction via Stratification of the Feature Space}
\begin{frame}{The Basics of Decision Trees}{Prediction via Stratification of the Feature Space}
    Roughly speaking, there are two steps for building a regression tree: \pause 

    \begin{enumerate}
        \item \textbf{Divide the predictor space} \\ \pause
        Divide the set of possible values of $X_1, \cdots, X_p$ into $J$ disticnt and non-overlapping regions, $R_1, \cdots, R_J$ \pause 

        \item \textbf{Make prediction for every $R_j$} \\ \pause 
        For every observation that falls into the region $R_j$ , we make the same prediction. \pause 
    \end{enumerate}

    The goal is to find $R_1, \cdots, R_J$ that minimize the $RSS$, given by \pause 

    \begin{equation*}
        \sum_{j=1}^J \sum_{i \in R_j} (y_i - \hat{y}_{R_j})^2
    \end{equation*}


\end{frame}

\begin{frame}{The Basics of Decision Trees}{Prediction via Stratification of the Feature Space}
    To construct $R_1, \cdots, R_J$ we use the \textbf{recursive binary splitting} approach: \pause  

    \begin{enumerate}
        \item \textbf{Define the \textit{region of predictor space}} \\ \pause 
        It's the space in which $X_j$ takes on a value less than the cutpoint $s$. \\ 
        For example, for $J=2$ \\

        \begin{align*}
            R_1 (j,s) = \{X|X_j < s\}, && R_2 (j, s) = \{X|X_j \geq s\}
        \end{align*} \pause 

        \item \textbf{Minimize RSS} \\ \pause 
        We seek the value $j$ and $s$ that minimize the equation, \pause 

        \begin{equation*}
            \sum_{i: x_i \in R_1(j,s)} (y_i - \hat{y}_{R_{1}} )^2 + \sum_{i: x_i \in R_2(j,s)} (y_i - \hat{y}_{R_{2}} )^2
        \end{equation*}

        \item \textbf{Repeat 1 and 2 in one of the previous identified regions} \pause

        \item \textbf{Continue until a stopping
criterion is reached} \\ 

    \end{enumerate}
\end{frame}

\begin{frame}{The Basics of Decision Trees}{Prediction via Stratification of the Feature Space}

    \begin{enumerate}
    \setcounter{enumi}{5}
        \item \textbf{Predict the response $y$ for $x$}\\ \pause  
        $y$ is the \textit{mean of the training observations} in the region to which $x$ belongs. \pause 
    \end{enumerate}
    For instance, 
    \begin{itemize}
        \item Suppose that in Step 1 we obtain two regions, $R_1$ and $R_2$. \pause 
        \item The \textit{response means} of the training observations in $R_1$ is 10, and  in $R_2$ is $20$. \pause 
        \item  For a given observation $X = x$, \\ \pause 
        $\rightarrow$ If $x \in R_1$, we will predict a value of $10$. \pause \\ 
        $\rightarrow$ If $x \in R_2$, we will predict a value of $20$. \pause
    \end{itemize}
    
\end{frame}


\subsection{Tree Pruning}
\begin{frame}{The Basics of Decision Trees}{Tree Pruning}

\begin{itemize}
    \item The process described is likely to overfit the data, leading to poor test set performance. \pause 
    \item This is because the resulting tree might be too complex. \pause 
    \item One way to solve this is consider a very large tree $T_0$ , and then prune it back in order to obtain a subtree that leads to the \textbf{lowest test error rate}. \pause 
    \item \textbf{Cost complexity pruning} —also known as \textit{weakest link pruning}—gives us a way to do just this.
    
\end{itemize}

\end{frame}

\begin{frame}{The Basics of Decision Trees}{Tree Pruning}
    \begin{itemize}
        \item  We consider a sequence of trees indexed by a tuning parameter $\alpha$. \pause
        
        \item For each value of $\alpha$ there corresponds a subtree $T \subset T_0$ such that \pause 
        \begin{equation*}
            \sum_{m=1}^{|T|} \sum_{i: x_i \in R_m} (y_i - \hat{y}_{R_m})^2 + \alpha |T|
        \end{equation*} 
    is as small as possible. \pause 
        Where, 
        \begin{itemize}
            \item $|T|$ indicates the number of terminal nodes of the tree $T$. \pause 
            \item $R_m$ is the subset of predictor space corresponding to the $m$th terminal node. \pause 
            \item $\hat{y}_{R_{m}}$ is the predicted response associated with $R_m$. \pause 
            \item $\alpha$ controls a trade-off between the subtree's complexity and its fit to the training data.
        \end{itemize}
    \end{itemize}


\end{frame}

\begin{frame}{The Basics of Decision Trees}{Tree Pruning}

For each value of $\alpha$ there corresponds a subtree $T \subset T_0$ such that \pause 
        \begin{equation}\label{eq:alpha_tree}
            \sum_{m=1}^{|T|} \sum_{i: x_i \in R_m} (y_i - \hat{y}_{R_m})^2 + \alpha |T|
        \end{equation} 
    is as small as possible. \pause 

\begin{itemize}
    \item When $\alpha = 0$, then $T = 
T_0$. \pause

    \item As $\alpha$ increases, the quantity (\ref{eq:alpha_tree}) will tend to be minimized for a smaller subtree. 

    \item We can select a value of $\alpha$ using cross-validation as is described next. 
\end{itemize}

\end{frame}

\begin{frame}{The Basics of Decision Trees}{Tree Pruning}
    \textbf{Algorithm for Building a Regression Tree}

    \begin{enumerate}
        \item Use recursive binary splitting to grow a large tree on the training data, stopping only when a threshold is reached. \pause 

        \item Use K-fold cross-validation to choose $\alpha$. That is, divide the training observations into $K$ folds. For each $k = 1, \cdots , K$: \pause

        \begin{enumerate}[a]
            \item Repeat step 1 on all but the $k$th fold of the training data. \pause

            \item Evaluate the mean squared prediction error on the data in the left-out kth fold, as a function of $\alpha$. \pause
        \end{enumerate}

        Average the results for each value of $\alpha$, and pick $\alpha$ to minimize the average error. \pause

        \item Return the subtree that corresponds to the chosen value of $\alpha$.
        
    \end{enumerate}
\end{frame}

\subsection{Classification Trees}
\begin{frame}{The Basics of Decision Trees}{Classification Trees}

\begin{itemize}
    \item We also use binary splitting to grow a classification tree. \pause
    \item We can use different criterias for making the binary splits \pause
\end{itemize}

\begin{enumerate}
    \item  \textbf{Classification error rate (CER)} \pause \\
    \begin{itemize}
        \item The plan is to assign and observation in a region $R_i$ to the \textit{most commonly occurring class} \pause

        \item Then, CER is the fraction of observations in the region $R_i$ that \textcolor{blue}{do not belong} to the \textit{most common class}. \pause 
        \begin{equation*}
        E = 1 - \max_k({\hat{p}_{mk}}).
        \end{equation*} \pause 

        $\rightarrow \hat{p}_{mk}$ is the proportion of observations in the $m$th region that are from the $k$th class. \pause 

        \item But, CER is not sufficiently sensitive for tree-growing. \pause
        
        \item In practice two other measures are preferable.
    \end{itemize}
    
\end{enumerate}

\end{frame}

\begin{frame}{The Basics of Decision Trees}{Classification Trees}

\begin{enumerate}
    \setcounter{enumi}{1}
    \item \textbf{Gini index} \pause 
    \begin{itemize}
        \item Is a measure of total variance across the $K$ classes. \pause 
        \begin{equation*}
            G = \sum_{k=1}^K \hat{p}_{mk}(1-\hat{p}_{mk}).
        \end{equation*} \pause
        \item $G$ is small if $\hat{p}_{mk} \rightarrow 1$ or $\hat{p}_{mk} \rightarrow 0$. \pause
        \item $G$ is referred to as a measure of \textit{node purity}: \pause a small value indicates that a node contains predominantly observations from a single class. \pause
    \end{itemize}

    \item \textbf{Entropy} \pause 
    \begin{itemize}
        \item Measures the impurity or uncertainty in a group of observations. \pause 
        \begin{equation*}
                D = - \sum_{k=1}^K \hat{p}_{mk} \log \hat{p}_{mk}.
        \end{equation*} \pause
        \item It can be shown that $D \rightarrow 0$ if all $\hat{p}_{mk} \rightarrow 0$ or $\hat{p}_{mk} \rightarrow 1$. \pause
        \item In this sense, Entropy is also referred as a measure of \textit{node purity}.
        
    \end{itemize}
\end{enumerate}

\end{frame}


\begin{frame}{The Basics of Decision Trees}{Classification Trees}

    \begin{itemize}
        \item When building a classification tree, either the Gini index or the entropy are typically used to evaluate the \textbf{quality} of a particular split. \pause 

        \item Any of these three approaches might be used when \textbf{pruning} the tree, but CER is preferable if \textbf{prediction accuracy} of the final pruned tree is the goal. 
    \end{itemize}

\end{frame}

\subsection{Advantages and Disadvantages of Trees}
\begin{frame}{The Basics of Decision Trees}{Advantages and Disadvantages of Trees}

\begin{columns}[T]
    \column{0.5\linewidth}
    \textbf{Advantages}
    \begin{itemize}
        \item[\checkmark] Trees can be displayed graphically, and are easily interpreted. \pause 

         \item[\checkmark] Can handle qualitative predictors without the need to create dummy variables. \pause 
    \end{itemize}


    \column{0.5\linewidth}
    \textbf{Disadvantages}
    \begin{itemize}
        \item[$\times$] Do not have the same level of predictive accuracy as some methods seen before. \pause 

         \item[$\times$] Suffers for high variance: a small change in the data can cause a large change in the final estimated tree. \pause 

    \end{itemize}
\end{columns}

\begin{block}{Note}
    By aggregating many decision trees, using methods like bagging, random forests, and boosting, the predictive performance of trees can be substantially improved.
\end{block}
    
\end{frame}