\begin{frame}{Shrinkage Methods}{Lasso Regression}

\begin{itemize}
    \item The lasso is an alternative that overcomes the main disadvantage of ridge regression. \pause 
    
    \item The lasso coefficients, $\hat{\beta}_\lambda^L$ , minimize the quantity \pause 

    \begin{equation} \label{eq:lasso}
        \sum_{i=1}^n (y_i - \beta_0 - \sum_{j=1}^p  \beta_j x_{ij} )^2 + \lambda \sum_{j=1}^p | \beta_j | = RSS + \lambda \sum_{j=1}^p | \beta_j |,
    \end{equation} \pause 

    \item As ridge regression, the lasso shrinks the coefficient estimates towards zero. \pause 
    
    \item However, this penalty has the effect of forcing some of the coefficient estimates to be \textbf{exactly equal to zero} when $\lambda$ is sufficiently large. \\ \pause 
    $\rightarrow$ Lasso performs variable selection. \pause 

    \item We say that the lasso yields \textit{sparse models}: models that involve only a subset of the variables. \pause

    \item Selecting a good value of $\lambda$ for the lasso is also critical. 
    
\end{itemize}
    
\end{frame}