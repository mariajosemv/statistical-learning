\begin{frame}{Shrinkage Methods}{Ridge Regression}

\begin{itemize}
    \item Recall that OLS fitting procedure estimates $\beta_0 , \beta_1 , \cdots , \beta_p$ using the values that minimize \pause
    
    \begin{equation}
        RSS = \sum_{i=1}^n (y_i - \beta_0 - \sum_{j=1}^p  \beta_j x_{ij} )^2
    \end{equation} \pause 

    \item \textit{Ridge regression} is very similar to OLS, except that the coefficients ridge are estimated by minimizing a slightly different quantity. \pause 

    \begin{equation*}
        \sum_{i=1}^n (y_i - \beta_0 - \sum_{j=1}^p  \beta_j x_{ij} )^2 + \lambda \sum_{j=1}^p \beta_j^2 = RSS + \lambda \sum_{j=1}^p \beta_j^2,
    \end{equation*} \pause 

    where $\lambda \geq 0$ is a \textit{tuning parameter} to be determined separately.
\end{itemize}
    
\end{frame}

\begin{frame}{Shrinkage Methods}{Ridge Regression}

        \begin{equation} \label{eq:ridge}
        \sum_{i=1}^n (y_i - \beta_0 - \sum_{j=1}^p  \beta_j x_{ij} )^2 + \lambda \sum_{j=1}^p \beta_j^2 = RSS + \lambda \sum_{j=1}^p \beta_j^2,
    \end{equation} \pause 

    Equation (\ref{eq:ridge}) trades off two different criteria: \pause
    
    \begin{enumerate}
        \item As with OLS, ridge regression seeks coefficient estimates that fit the data well, by making the RSS small. \pause 

        \item The term \textcolor{blue}{$ \lambda \sum_{j=1}^p \beta_j^2$} is the \textit{shirkage penalty}.  \\ \pause 
        $\rightarrow$ is small when $\beta_0 , \beta_1 , \cdots , \beta_p$ are close to zero. \pause \\ 
        $\rightarrow$ It has the effect of shrinking $\beta_j$ towards zero. \pause     
    \end{enumerate}

    $\lambda$ serves to control the impact of these two terms on the coefficient estimates. \pause 

    \begin{itemize}
        \item When $\lambda = 0$, the penalty term has no effect. The results are the OLS estimates. \pause 
        \item When $\lambda \rightarrow \infty$, the impact of the shrinkage penalty grows, and the ridge regression coefficient estimates will approach zero.
    \end{itemize}
    
\end{frame}

\begin{frame}{Shrinkage Methods}{Ridge Regression}
    \begin{itemize}
        \item OLS generates only one set of coefficient estimates. \pause 
        
        \item But ridge regression will produce a different set of coefficient estimates, $\hat{\beta}^R_\lambda$, for each value of $\lambda$. \pause 
        
        \item Selecting a good value for $\lambda$ is critical; we defer this discussion later, where we use cross-validation. \pause 

        \item The disadvantage is that ridge will include all $p$ predictors in the final model. \pause 
        
        \item The penalty $\lambda \sum \beta_j^2$ in (\ref{eq:ridge}) will shrink all of the coefficients \textbf{towards zero}, but it will not set any of them exactly to zero (unless $\lambda = \infty$). \pause 
        
        \item This may not be a problem for prediction accuracy. \pause 
        
        \item But it can create a challenge in model interpretation.
    \end{itemize}
\end{frame}