\begin{frame}{Introduction}

\begin{itemize}
    \item In the regression setting, the standard linear model \pause
    $$Y = \beta_0 + \beta_1 X + \cdots + \beta_p X_p + \epsilon$$

    is typically fitted using least squares. \pause

    \item But as we will see, OLS procedures can have problems with \textbf{prediction accuracy} and \textbf{model interpretability}.
    
\end{itemize}


\end{frame}

\begin{frame}{Introduction}

\textbf{Prediction Accuracy:} \pause 

\begin{itemize}
    \item If the true relationship between the $Y$ and $X$ is approximately linear, OLS fit will have low bias. \pause 
    
    \item If $n \gg p$, then OLS fit tend to also have low variance, and hence will perform well on test observations. \pause 

    \item If $n \not \gg p$ then there can be a lot of variability in the OLS fit. \pause 
    \\ $\rightarrow$ This results in overfitting and poor predictions on test set. \pause 

    \item If $n < p$, then there is no longer a unique OLS: the variance is infinite so the method cannot be used at all. \pause 
\end{itemize}

\textbf{Model Interpretability:} \pause 

\begin{itemize}
    \item It is often the case that some variables used in a multiple regression model are in fact not associated with $Y$. \pause 

    \item Including such variables leads to complexity in the resulting model. \pause 
    
    \item Removing these variables we lead to a model that is more easily interpreted. \pause 
    
    \item With OLS is extremely unlikely to yield any coefficient estimates that are exactly zero. 
\end{itemize}


\end{frame}

\begin{frame}{Introduction}

We'll discuss two important alternatives methods to using OLS: \pause 

\begin{enumerate}
    \item \textbf{Shrinkage} \pause  
    \begin{itemize}
        \item Fit a model involving all $p$ predictors. \pause 

        \item Estimated coefficients are shrunken towards zero. \pause 

        \item This shrinkage has the effect of reducing variance \pause 
    \end{itemize}

    \item \textbf{Dimension reduction} \pause 
    \begin{itemize}
        \item Projecting the $p$ predictors into an $M-$dimensional subspace, where $M < p$. \pause 

        \item This is done by computing $M$ different linear combinations, or projections, of the variables. \pause 
        
        \item Use these projections as predictors in OLS. 
    \end{itemize}
    
\end{enumerate}

    
\end{frame}






