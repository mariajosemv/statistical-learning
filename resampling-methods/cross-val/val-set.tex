\begin{frame}{Cross-validation}{The validation set approach}
    
\begin{enumerate}
    \item Randomly divide the available set of observations into two parts: a training set and a validation set. \pause
    \item Fit the model on the training set. \pause
    \item Predict the responses for the observations in the validation set. \pause
    \item The resulting validation set error rate provides an estimate of the test error rate. \pause
\end{enumerate}

\begin{block}{\textbf{Drawbacks:}}
\begin{itemize}
    \item The validation estimate of the test error rate can be highly variable:  \pause
    \\ $\rightarrow$ Random splitting.  \pause

    \item Statistical methods tend to perform worse when trained on fewer observations: \pause \\  The validation set error rate may tend to overestimate the test error rate for the entire data set. \pause 
\end{itemize}

    
\end{block}

We will present cross-validation, a method that addresses these two issues. 
\end{frame}