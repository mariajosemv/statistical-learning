\begin{frame}{Cross-Validation}{Bias-variance trade-off for k-fold cross-validation}

Summarizing, 

\begin{itemize}
    \item Validation set approach can lead to overestimates of the test error rate. \pause \\
    $\rightarrow$ The training set used to fit the model contains only half the observations of the entire data set. \pause 

    \item LOOCV will give approximately unbiased estimates of the test error. \pause \\
    $\rightarrow$ Each training set contains $n - 1$ observations, almost as many as the full number of observations. \pause

    \item k-fold CV for $k = 5$ or $k = 10$ will lead to an intermediate level of bias. \pause \\ 
    $\rightarrow$ Each training set contains approximately $(k - 1)n/k$ observations. \pause 

\end{itemize}


\textcolor{blue}{For bias reduction, LOOCV is to be preferred to k-fold CV.} 

\end{frame}

\begin{frame}{Cross-Validation}{Bias-variance trade-off for k-fold cross-validation}
    
However, there is some problems with LOOCV: 

\begin{itemize}
    \item Has higher variance than does k-fold CV with $k < n$. \pause 
    \item We are using almost identical training observations to then average the outputs. \pause \\ 
    \item These outputs are highly correlated with each other! \pause 

\end{itemize}

In contrast, when using k-fold CV with $k < n$, \pause 

\begin{itemize}
    \item We split the data into folds and then average the outputs. \pause 
    \item The outputs are somewhat less correlated with each other. \pause 
\end{itemize}

The mean of \textcolor{blue}{many highly correlated} quantities has higher variance than does the mean of many quantities that are not
as highly correlated. \pause 

\begin{block}{Bias-variance trade-off}
    \begin{itemize}
        \item Given these considerations, one performs k-fold cross-validation using $k = 5$ or $k = 10$. \pause 
        \item These values have been shown to yield test error rate that suffer neither from excessively high bias nor from very high variance.
    \end{itemize}
\end{block}


    
\end{frame}